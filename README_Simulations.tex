% Options for packages loaded elsewhere
\PassOptionsToPackage{unicode}{hyperref}
\PassOptionsToPackage{hyphens}{url}
%
\documentclass[
]{article}
\usepackage{amsmath,amssymb}
\usepackage{lmodern}
\usepackage{iftex}
\ifPDFTeX
  \usepackage[T1]{fontenc}
  \usepackage[utf8]{inputenc}
  \usepackage{textcomp} % provide euro and other symbols
\else % if luatex or xetex
  \usepackage{unicode-math}
  \defaultfontfeatures{Scale=MatchLowercase}
  \defaultfontfeatures[\rmfamily]{Ligatures=TeX,Scale=1}
\fi
% Use upquote if available, for straight quotes in verbatim environments
\IfFileExists{upquote.sty}{\usepackage{upquote}}{}
\IfFileExists{microtype.sty}{% use microtype if available
  \usepackage[]{microtype}
  \UseMicrotypeSet[protrusion]{basicmath} % disable protrusion for tt fonts
}{}
\makeatletter
\@ifundefined{KOMAClassName}{% if non-KOMA class
  \IfFileExists{parskip.sty}{%
    \usepackage{parskip}
  }{% else
    \setlength{\parindent}{0pt}
    \setlength{\parskip}{6pt plus 2pt minus 1pt}}
}{% if KOMA class
  \KOMAoptions{parskip=half}}
\makeatother
\usepackage{xcolor}
\usepackage[margin=1in]{geometry}
\usepackage{graphicx}
\makeatletter
\def\maxwidth{\ifdim\Gin@nat@width>\linewidth\linewidth\else\Gin@nat@width\fi}
\def\maxheight{\ifdim\Gin@nat@height>\textheight\textheight\else\Gin@nat@height\fi}
\makeatother
% Scale images if necessary, so that they will not overflow the page
% margins by default, and it is still possible to overwrite the defaults
% using explicit options in \includegraphics[width, height, ...]{}
\setkeys{Gin}{width=\maxwidth,height=\maxheight,keepaspectratio}
% Set default figure placement to htbp
\makeatletter
\def\fps@figure{htbp}
\makeatother
\setlength{\emergencystretch}{3em} % prevent overfull lines
\providecommand{\tightlist}{%
  \setlength{\itemsep}{0pt}\setlength{\parskip}{0pt}}
\setcounter{secnumdepth}{-\maxdimen} % remove section numbering
\ifLuaTeX
  \usepackage{selnolig}  % disable illegal ligatures
\fi
\IfFileExists{bookmark.sty}{\usepackage{bookmark}}{\usepackage{hyperref}}
\IfFileExists{xurl.sty}{\usepackage{xurl}}{} % add URL line breaks if available
\urlstyle{same} % disable monospaced font for URLs
\hypersetup{
  pdftitle={Code for dynamic mixture of experts models},
  hidelinks,
  pdfcreator={LaTeX via pandoc}}

\title{Code for dynamic mixture of experts models}
\usepackage{etoolbox}
\makeatletter
\providecommand{\subtitle}[1]{% add subtitle to \maketitle
  \apptocmd{\@title}{\par {\large #1 \par}}{}{}
}
\makeatother
\subtitle{TCH-21-168.R1}
\author{}
\date{\vspace{-2.5em}}

\begin{document}
\maketitle

\hypertarget{description}{%
\subsection{Description}\label{description}}

The provided code files implement all simulation experiments presented
in the Table 2, Section 5. The table presents five experiments:

\begin{itemize}
\tightlist
\item
  M1: Static Poisson regression data generating process model.
\item
  M2: Dynamic Poisson regression data generating process model.
\item
  M3: Dynamic mixture of Poisson regression experts data generating
  process model.
\item
  M4: Static mixture of Poisson autoregressive experts data generating
  process model.
\item
  M5: Dynamic mixture of Poisson autoregressive experts data generating
  process model.
\end{itemize}

In the experiments M4 and M5, the SMC inference methodology proposed in
the paper is compared with the MCMC method in Villani et al.~(2012). The
SMC inference is implemented in R in the form of a package; the R
package's name is \texttt{dmoe}. The MCMC inference is implemented in
Matlab which is also used to simulate data (in M4 and M5 experiments)
and to train models with static parameter. The other experiments, M1-M3,
are fully implemented in R.

\hypertarget{what-is-included-in-the-submited-files}{%
\subsection{What is included in the submited
files}\label{what-is-included-in-the-submited-files}}

We submit:

\begin{itemize}
\tightlist
\item
  The R code for the simulation experiments in the folder
  \texttt{simulations}.

  \begin{itemize}
  \tightlist
  \item
    The R package \texttt{dmoe} zip file \texttt{dmoe.zip} is saved in
    \texttt{simulations}.
  \end{itemize}
\item
  The Matlab code for the MCMC sampler and simulation experiments M4-M5
  in the folder \texttt{OnlineMoEMatlab}.
\end{itemize}

\hypertarget{r-code-instructions}{%
\subsection{R code instructions}\label{r-code-instructions}}

The \texttt{simulations} folder includes:

\begin{itemize}
\tightlist
\item
  The \texttt{R} folder. All source codes implementing the simulation
  experiments are saved in this folder.
\item
  The \texttt{data} folder. In case one can not run the Matlab code, two
  csv files \texttt{simdata\_static.csv} and \texttt{simdata\_dyn.csv},
  which contains \(50\) datasets generated from M4 and M5 respectively,
  are saved in this folder.
\item
  The \texttt{results} folder. All results are saved in this folder. In
  this folder we also saved files containing the log predictive scores
  (LPS), computed from the MCMC sampler, required to replicate the
  Figure 5.3.
\item
  The \texttt{plots} folder. It contains plots and the code generating
  the plots.
\end{itemize}

The \texttt{R} folder includes the file
\texttt{run\_all\_simulation\_experiments.R} which automates all
experiments. The code in this file does:

\begin{itemize}
\tightlist
\item
  Install the \texttt{dmoe} package
\item
  Install all libraries required to run the simulation experiments and
  to generate the plots.
\item
  Run each of the M1-M4 experiments and the generate the plots
  automatically.
\end{itemize}

To run the code in \texttt{run\_all\_simulation\_experiments.R}:

\begin{itemize}
\tightlist
\item
  If you use Rstudio, double click on the \texttt{simulations.Rproject}.
  It will open up an Rstudio window with all files.\\
\item
  Otherwise, set the working directory to the \texttt{simulations}
  folder and run the code in
  \texttt{run\_all\_simulation\_experiments.R}.
\end{itemize}

Note: - All lines with the \texttt{source("path\_to\_file.R")} runs
automatically an experiment. All are commented out except the
experiments M1-M3. In case you want to run an experiment other than
M1-M3, just uncomment it and run it. - The experiments take a long time
to execute.To reduce the running time, I have reduced the iterations in
each experiment to 10 iterations. With 10 iterations each experiment
takes roughly 2 hours.

\hypertarget{matlab-code-instructions}{%
\subsection{Matlab code instructions}\label{matlab-code-instructions}}

\begin{itemize}
\item
  Put the folder \texttt{OnlineMoEMatlab} and its sub-folders on
  Matlab's path.
\item
  Run the file \texttt{SimulateEstimateAndComputeLPS.m} in the folder
  \texttt{OnlineMoEMatlab/projects/onlineMoE}
\item
  The code in \texttt{SimulateEstimateAndComputeLPS.m} does:

  \begin{itemize}
  \tightlist
  \item
    Generates data from static and dynamic mixture of Poisson
    autoregressive experts.
  \item
    Gibbs sampling for mixture of Poisson autoregressive experts.
  \item
    Prediction on test data where the posterior is updated at every
    \texttt{batchSize} observation.
  \item
    Compute log predictive score (LPS) on the test data.
  \end{itemize}
\end{itemize}

Note: the files \texttt{OnlineMoEMatlab/GSM.m} and
\texttt{OnlineMoEMatlab/models/pois/Pois.m} Contains a lot of MCMC and
models settings.

\end{document}
